% interactcadsample.tex
% v1.03 - April 2017

\documentclass[]{interact}

\usepackage{epstopdf}% To incorporate .eps illustrations using PDFLaTeX, etc.
\usepackage{subfigure}% Support for small, `sub' figures and tables
%\usepackage[nolists,tablesfirst]{endfloat}% To `separate' figures and tables from text if required

\usepackage{natbib}% Citation support using natbib.sty
\bibpunct[, ]{(}{)}{;}{a}{}{,}% Citation support using natbib.sty
\renewcommand\bibfont{\fontsize{10}{12}\selectfont}% Bibliography support using natbib.sty

\theoremstyle{plain}% Theorem-like structures provided by amsthm.sty
\newtheorem{theorem}{Theorem}[section]
\newtheorem{lemma}[theorem]{Lemma}
\newtheorem{corollary}[theorem]{Corollary}
\newtheorem{proposition}[theorem]{Proposition}

\theoremstyle{definition}
\newtheorem{definition}[theorem]{Definition}
\newtheorem{example}[theorem]{Example}

\theoremstyle{remark}
\newtheorem{remark}{Remark}
\newtheorem{notation}{Notation}

% see https://stackoverflow.com/a/47122900
\usepackage{color}
\usepackage{fancyvrb}
\newcommand{\VerbBar}{|}
\newcommand{\VERB}{\Verb[commandchars=\\\{\}]}
\DefineVerbatimEnvironment{Highlighting}{Verbatim}{commandchars=\\\{\}}
% Add ',fontsize=\small' for more characters per line
\usepackage{framed}
\definecolor{shadecolor}{RGB}{248,248,248}
\newenvironment{Shaded}{\begin{snugshade}}{\end{snugshade}}
\newcommand{\AlertTok}[1]{\textcolor[rgb]{0.94,0.16,0.16}{#1}}
\newcommand{\AnnotationTok}[1]{\textcolor[rgb]{0.56,0.35,0.01}{\textbf{\textit{#1}}}}
\newcommand{\AttributeTok}[1]{\textcolor[rgb]{0.77,0.63,0.00}{#1}}
\newcommand{\BaseNTok}[1]{\textcolor[rgb]{0.00,0.00,0.81}{#1}}
\newcommand{\BuiltInTok}[1]{#1}
\newcommand{\CharTok}[1]{\textcolor[rgb]{0.31,0.60,0.02}{#1}}
\newcommand{\CommentTok}[1]{\textcolor[rgb]{0.56,0.35,0.01}{\textit{#1}}}
\newcommand{\CommentVarTok}[1]{\textcolor[rgb]{0.56,0.35,0.01}{\textbf{\textit{#1}}}}
\newcommand{\ConstantTok}[1]{\textcolor[rgb]{0.00,0.00,0.00}{#1}}
\newcommand{\ControlFlowTok}[1]{\textcolor[rgb]{0.13,0.29,0.53}{\textbf{#1}}}
\newcommand{\DataTypeTok}[1]{\textcolor[rgb]{0.13,0.29,0.53}{#1}}
\newcommand{\DecValTok}[1]{\textcolor[rgb]{0.00,0.00,0.81}{#1}}
\newcommand{\DocumentationTok}[1]{\textcolor[rgb]{0.56,0.35,0.01}{\textbf{\textit{#1}}}}
\newcommand{\ErrorTok}[1]{\textcolor[rgb]{0.64,0.00,0.00}{\textbf{#1}}}
\newcommand{\ExtensionTok}[1]{#1}
\newcommand{\FloatTok}[1]{\textcolor[rgb]{0.00,0.00,0.81}{#1}}
\newcommand{\FunctionTok}[1]{\textcolor[rgb]{0.00,0.00,0.00}{#1}}
\newcommand{\ImportTok}[1]{#1}
\newcommand{\InformationTok}[1]{\textcolor[rgb]{0.56,0.35,0.01}{\textbf{\textit{#1}}}}
\newcommand{\KeywordTok}[1]{\textcolor[rgb]{0.13,0.29,0.53}{\textbf{#1}}}
\newcommand{\NormalTok}[1]{#1}
\newcommand{\OperatorTok}[1]{\textcolor[rgb]{0.81,0.36,0.00}{\textbf{#1}}}
\newcommand{\OtherTok}[1]{\textcolor[rgb]{0.56,0.35,0.01}{#1}}
\newcommand{\PreprocessorTok}[1]{\textcolor[rgb]{0.56,0.35,0.01}{\textit{#1}}}
\newcommand{\RegionMarkerTok}[1]{#1}
\newcommand{\SpecialCharTok}[1]{\textcolor[rgb]{0.00,0.00,0.00}{#1}}
\newcommand{\SpecialStringTok}[1]{\textcolor[rgb]{0.31,0.60,0.02}{#1}}
\newcommand{\StringTok}[1]{\textcolor[rgb]{0.31,0.60,0.02}{#1}}
\newcommand{\VariableTok}[1]{\textcolor[rgb]{0.00,0.00,0.00}{#1}}
\newcommand{\VerbatimStringTok}[1]{\textcolor[rgb]{0.31,0.60,0.02}{#1}}
\newcommand{\WarningTok}[1]{\textcolor[rgb]{0.56,0.35,0.01}{\textbf{\textit{#1}}}}

% Pandoc citation processing

\usepackage{amsmath}
\usepackage{bm}
\usepackage{booktabs}
\usepackage{hyperref}
\usepackage[utf8]{inputenc}
\def\tightlist{}


\begin{document}

\articletype{TEACHER'S CORNER}

\title{A tutorial for combining static and dynamic panel models in
structural equation modeling: A guide to current panel models with
observed and latent variables}


\author{\name{Henrik Kenneth Andersen$^{a}$, Jochen Mayerl$^{a}$, Elmar
Schlüter$^{b}$}
\affil{$^{a}$Chemnitz University of Technology, Thüringer Weg 9, 09126
Chemnitz, DE; $^{b}$Justus-Liebig-Universität Gießen, Karl-Glöckner-Str.
21E, 35394 Gießen, DE}
}

\thanks{CONTACT Henrik Kenneth
Andersen. Email: \href{mailto:henrik.andersen@soziologie.tu-chemnitz.de}{\nolinkurl{henrik.andersen@soziologie.tu-chemnitz.de}}, Jochen
Mayerl. Email: \href{mailto:jochen.mayerl@soziologie.tu-chemnitz.de}{\nolinkurl{jochen.mayerl@soziologie.tu-chemnitz.de}}, Elmar
Schlüter. Email: \href{mailto:elmar.schlueter@sowi.uni-giessen.de}{\nolinkurl{elmar.schlueter@sowi.uni-giessen.de}}}

\maketitle

\begin{abstract}
This template is for authors who are preparing a manuscript for a Taylor
\& Francis journal using the \LaTeX~document preparation system and the
`interact\} class file, which is available via selected journals' home
pages on the Taylor \& Francis website.
\end{abstract}

\begin{keywords}
Panel SEM; dynamic panel model; random-intercept cross-lagged panel
model; structured residuals; measurement invariance
\end{keywords}

\hypertarget{introduction}{%
\section{Introduction}\label{introduction}}

Panel data, i.e., repeated measures of the same observational units over
time, are becoming increasingly popular in the social and behavioural
sciences. While more elaborate and expensive in terms of data
collection, they offer a variety of benefits over cross-sectional data,
such as the opportunity to establish temporal precedence, and increased
statistical power due to the typically larger pooled sample size
\citep{Curran2011}. However, the most attractive aspect of panel data is
the ability they afford to examine causal relations in a more rigorous
test of substantive theories. Namely, panel data allow one to decompose
the typical regression error term into a part that is constant over time
within units, and a part that changes over time. The part that does not
change can be seen as the combined effect of all time-invariant
characteristics, such as personality traits, sex, place of birth, etc.
Various techniques are available to identify and statistically control
for this time-invariant error component which are typically referred to
under the banner of \emph{random} and \emph{fixed effects}. These stable
characteristics mean that the assumption of independent observations is
typically violated with panel data. While random effects models
explicitly model dependency to achieve unbiased standard errors and
significance testing, fixed effects regression goes a step further and
statistically controls for confounding between the model covariates and
\emph{all} time-invariant potential confounders. Thus, fixed effects
regression allows researchers to identify causal effects under less
restrictive assumptions. Such random and fixed effects models which
account for stable between-unit differences, i.e., \emph{unobserved
heterogeneity}, are referred to as \emph{static} panel models.

Structural equation modeling (SEM) is a flexible regression framework
with which a large variety of panel regression models can be estimated.
Random and fixed effects regression is easily implemented in SEM
\citep{Bollen2010}. In fact, it may not yet be common knowledge, but one
of the most popular static panel regression models in SEM, the latent
curve model \citep[LCM, which goes by a number of names, such as the
latent growth curve, see][]{Meredith1990} is essentially a fixed effects
model that can control for unobserved stable differences not just in
level, but also growth \citep{Teachman2001, Teachman2014}. One of the
biggest advantages of SEM compared to conventional (e.g., least squares)
approaches is the ability to model the constructs of interest as latent
variables, thus accounting for measurement error. Further, measurement
invariance testing checks whether the underlying measurement model is
time-invariant, or if the correlations between the indicators change
systematically over time. If the latter is the case, then it would not
be appropriate to compare levels of the latent variables over time or
the regression effects between them. I.e., if the construct of interest
has changed in its composition, then we cannot say that, say, attitudes
are becoming more or less positive over time; the attitudes at time
point one may not even be comparable to the attitudes at time point two.
Besides that, SEM allows for a minute testing of a range of other model
assumptions \citep[e.g., constant effects over time, contemporary
vs.~strict vs.~sequential exogeneity,][]{Bollen2010, Bruederl2015}.

SEM also offers a straightforward approach to \emph{dynamic} panel
models, i.e., those in which the lagged dependent variable is included
as a predictor for the current dependent variable
\citep{Zyphur2019a, Zyphur2019b}. Dynamic panel models can account for
processes in which there is a theoretical expectation that previous
realizations of a variable should causally affect later realization
\citep[i.e., state dependence,][]{Heckman1981, Hsiao2014}, or for those
in which the inclusion of the lagged dependent variable might be done
out of pragmatic considerations, e.g., to control for potential
confounding by other unobserved lagged variables \citep{Kuehnel2019}.

Recently, models combining both \emph{static} and \emph{dynamic}
components have become popular in panel SEM. For example, the
Autoregressive Latent Trajectory Model
\citep[ALT,][]{Bollen2004, Curran2001} and Dynamic Panel Model
\citep[DPM,][]{Allison2017, Williams2018, Moral2019} combine
autoregressive and random/fixed effects models to account for both
unobserved heterogeneity and state dependence. Another increasingly
popular approach can be seen in the Random Intercept Cross-Lagged Panel
Model
\citep[RI-CLPM,][]{Hamaker2015, Mulder2020, Zyphur2019a, Zyphur2019b}
and the Latent Curve Model with Structured Residuals
\citep[LCM-SR,][]{Curran2014}. These models are re-expressions of the
typical DPM and ALT, respectively, that model the autoregressive and
covariate effects at the \emph{residual-level}, i.e., what is left over
after regressing the observed (or latent variables) on the individual
effects. These models offer an attractive intuitive logic: the residuals
in these models represent the difference between the observation and the
individual effects (representing the stable characteristics). These are
essentially the \emph{demeaned} (or \emph{detrended} in the case of the
LCM-SR) versions of the observed variables, i.e., the stable error
component is subtracted from the observation thereby re-expressing it in
terms of its deviation from the stable per-unit overall average (or
trajectoy). As such, the between variance is eliminated and the effects
of interest are, it follows, purely \emph{within-unit}
\citep{Hamaker2015}.

This article is meant as a tutorial for estimating the ALT, DPM, LCM-SR
and RI-CLPM in \texttt{R} and \texttt{Mplus}, two of the currently most
popular SEM software packages. We demonstrate both one- and two-sided
versions (in one-sided models, one of variables is treated as the
`definitive' dependent variable, while in two-sided models the focus is
more on examining reciprocal relations in which both or all variables
are at some point dependent on other covariates) of each of these models
with observed indicators and latent variables to account for measurement
error. Further, we demonstrate longitudinal measurement invariance
testing for continuous and ordinal observed indicators. We also briefly
touch on the opportunity to relax assumptions concerning time-invariant
effects, error variances and the exogeneity of the model covariates. We
use simulated data (found in the supplementary materials) for the sake
of didactic simplicity, and in order to provide a `toy' dataset for
researchers to experiment with. The article assumes knowledge of the
basic syntax of both \texttt{R} and \texttt{Mplus}. It describes how to
estimate the discussed models, but does not the operators and language
itself (\texttt{ON}, \texttt{BY}, \texttt{WITH},
\texttt{\textasciitilde{}}, \texttt{=\textasciitilde{}},
\texttt{\textasciitilde{}\textasciitilde{}}, etc.).

\hypertarget{background}{%
\section{Background}\label{background}}

xxx

\hypertarget{data}{%
\section{Data}\label{data}}

xxx

\begin{table}
\begin{tabular}{l c c c}
\toprule
    & Effects      & Factor loadings & Data \\
\midrule
df1 & Constant     & Constant        & Continuous \\
df2 & Time-varying & Constant        & Continuous \\
df3 & Time-varying & Time-varying    & Continuous \\
df4 & Time-varying & Time-varying    & Ordinal \\
\bottomrule
\end{tabular} \vspace{3pt}
\caption{xxx} 
\label{xxx}
\end{table}

\begin{Shaded}
\begin{Highlighting}[]
\KeywordTok{library}\NormalTok{(lavaan)}
\NormalTok{df1 \textless{}{-}}\StringTok{ }\KeywordTok{readRDS}\NormalTok{(}\StringTok{"data/df{-}allison{-}sim.rda"}\NormalTok{)}
\end{Highlighting}
\end{Shaded}

\hypertarget{models}{%
\section{Models}\label{models}}

\hypertarget{observation-level-models}{%
\subsection{Observation-level models}\label{observation-level-models}}

The DPM and ALT are considered `observation-level' models in which the
autoregressive and covariate (e.g., cross-lagged) effects are specified
between either the observed variables or latent variables representing
the measurement error adjusted constructs of interest, see Figure xxx(a)
and xxx(b) for two-sided representations.

A two-sided DPM with observed variables can be expressed as
\begin{align}
\begin{split}
y_{it} & = \rho_{t} y_{it-1} + \beta_{t} x_{it-1} + \eta_{1i} + \varepsilon_{it}, \\
x_{it} & = \varphi_{t} x_{it-1} + \gamma_{t} y_{it-1} + \alpha_{1i} + \delta_{it}, \ t = 1, \ldots, T \label{dpm}
\end{split}
\end{align} where \(i = 1, \ldots, N\) and \(t = 0, \ldots, T\), and
\(y_{it}\) and \(x_{it}\) are observed variables of interest, and
\(\eta_{1i}\) and \(\alpha_{1i}\) are latent variables representing the
combined effect of all stable characteristics and \(\varepsilon_{it}\)
and \(\delta_{it}\) are the idiosyncratic errors for \(y\) and \(x\),
respectively. \(\rho_{t}\) and \(\varphi_{t}\) are the autoregressive
effects and \(\beta_{t}\) and \(\gamma_{t}\) are the cross-lagged
effects at time \(t\).

For the sake of simplicity, we assume here and throughout that the
observed variables are mean-centered before the analysis. Conditional on
the model covariates and individual effects, we assume the temporal
errors are independent within units, i.e.,
\(E(\varepsilon_{it}\varepsilon_{is} | \eta_{i}, \bm{x}_{i}) = 0\) and
\(E(\delta_{it}\delta_{is} | \alpha_{i}, \bm{y}_{i})=0, \ t \ne s\),
where \(\bm{x}_{i} = (x_{i0}, \ldots, x_{iT})\) and
\(\bm{y}_{i} = (y_{i0}, \ldots, y_{iT})\). This is the strict exogeneity
assumption that can also be expressed as
\(E(\varepsilon_{it}\bm{x}_{i}) = \bm{0}\),
\(E(\delta_{it}\bm{y}_{i}) = \bm{0}\), which is stronger than the
contemporary exogeneity assumption \(E(\varepsilon_{it}x_{it}) = 0\) and
\(E(\delta_{it}y_{it}) = 0\) typical to random effects models
\citep{Bruederl2015, Wooldridge2012}. The strict exogeneity assumption
is in line with the simulated DGP and it would be unusual to assume the
errors at one point in time should be correlated with the covariates at
other points in time without a strong theoretical argument. For the sake
of parsimony, we assume constant effects over time, i.e.,
\(\rho_{t} = \rho\), \(\beta_{t} = \beta\) and so forth. With sufficient
degrees of freedom, a different coefficient per time point could be
estimated.

One important thing to keep in mind is that the residual-level models
(RI-CLPM, LCM-SR) are re-expressions of \emph{constrained} versions of
their observation-level counterparts (DPM, ALT, respectively). I.e.,
constrained panel models place specific assumptions on the initial
conditions; in other words the way in which the latent individual
effects influence the \(t = 0\) variables. Specifically, they assume
that the dynamic process is \emph{stationary}, i.e., the autoregressive
effect is less than one in absolute value and has been going on long
enough to have reached \emph{equilibrium}, i.e., the means and
covariances of the variables of interest are no longer changing over
time
\citep{Ou2016, Curran2001, Bollen2004, Jongerling2011, Andersen2021}.
These are potentially strict assumptions and so the residual-level
models, just like the constrained observation-level counterparts, are
not appropriate for modelling some dynamic processes. The initial
observations in the observation-level models, on the other hand, can be
treated as either constrained or \emph{predetermined}. Predetermined
models place no specific assumptions on the initial conditions, instead
treating the \(t = 0\) variables as exogenous, allowing them to covary
freely with the individual effects and other exogenous covariates. This
makes the predetermined models more flexible and able to model a wider
range of dynamic processes, \emph{at the cost of parsimony}: treating
the initial observations as exogenous means a number of covariances must
be estimated rather than fixed to specific values. For this reason,
specify predetermined observation-level models by treating the first
time point as exogenous.

The so-called bivariate predetermined ALT is essentially the same as the
above-outlined DPM that further controls for unit heterogeneity in terms
of \emph{trajectory}. \begin{align}
\begin{split}
y_{it} & = \rho_{t} y_{it-1} + \beta_{t} x_{it-1} + \eta_{1i} + t\eta_{2i} + \varepsilon_{it}, \\
x_{it} & = \varphi_{t} x_{it-1} + \gamma_{t} y_{it-1} + \alpha_{1i} + t\alpha_{2i} + \delta_{it}, \ t = 1, \ldots, T \label{alt}
\end{split}
\end{align} where \(\eta_{2i}\) and \(\alpha_{2i}\) are the unit
specific trajectories multiplied by \(t\) for linear growth.\footnote{Other
  functions of time are also possible to model, either by adding another
  individual effects variable for polynomial functions, or by allowing
  the coefficient of the trajectories to be estimated freely for
  \(t > 1\).} Essentially, we assume that individuals differ not only in
regards to their stable overall levels, but also trajectories or slopes.
In the multilevel literature, this is referred to as a random intercept,
random slope model, \citet{Hox2010}. However, while the typical random
effects/multilevel model with random intercepts and slopes assumes the
individual trajectories are unrelated with the other model covariates,
we can easily control for the possibility that they are related (a much
more plausible assumption) by allowing the intercept and slope factors
to covary with the covariates; thereby moving from random effects
assumptions (e.g.,
\(E(\eta_{1i}\bm{x}_{i}) = E(\eta_{2i}\bm{x}_{i}) = \bm{0}\), where the
same applies to the individual effects of \(x_{it}\), \(\alpha_{i}\),
and \(\bm{y}_{i}\)) to fixed effects ones
(\(E(\eta_{1i}\bm{x}_{i}) \ne \bm{0}\),
\(E(\eta_{2i}\bm{x}_{i}) \ne \bm{0}\)).

For a DPM or ALT in which the constructs of interest are modelled as
multiple indicator latent variables, we must add a measurement model
portion to the model \begin{align}
\begin{split}
y_{jit} & = \lambda^{y}_{jt}y_{it} + \varepsilon_{jit}, \\
x_{jit} & = \lambda^{x}_{jt}x_{it} + \delta_{jit}, \label{meas-model}
\end{split}
\end{align} with \(y_{jit}\) and \(x_{jit}\) as the \(j\)th indicators,
\(j = 1, \ldots, K\), and where \(\lambda^{y}_{jt}\) and
\(\lambda^{x}_{jt}\) are the factor loadings of the \(j\)th indicator at
time \(t\) on the latent variables \(y_{it}\) and \(x_{it}\),
respectively.

\hypertarget{model-specification}{%
\subsubsection{Model specification}\label{model-specification}}

There are four `blocks' of code necessary to specify a DPM or an ALT
with multi-indicator latent time-varying variables.\\

\begin{enumerate}
\item Measurement models
\item Individual effects
\item Regressions
\item Covariances 
\end{enumerate}

In the first block, the \emph{measurement models} we specify the latent
variables for the constructs of interest (lines 2--10). In this example,
both the independent and dependent variables, \(x\) and \(y\), are
measured using three separate indicators (\texttt{x1t}, \texttt{x2t},
\texttt{x3t}, \texttt{y1t}, \texttt{y2t}, \texttt{y3t}) at four discrete
points in time. Note that no constraints have been placed on the factor
loadings. The default behaviour of \texttt{lavaan} is to fix the factor
loading for the first indicator (\texttt{x1t}, \texttt{y1t}) to one.

Next, in lines 11--13, we specify the individual effects to account for
individual heterogeneity. Here, we use two latent variables,
\texttt{alpha} and \texttt{eta} for each the independent and dependent
variable, respectively. Note that the individual effects point towards
the newly created latent constructs. All of the factor loadings are
fixed to one. This represents the belief that the effect of the stable
characteristics are constant over time \citep{Bollen2010}.

In lines 14--20, we specify the regressions for the autoregressive and
cross-lagged effects. Note here that we use the same labels
(\texttt{phi}, \texttt{rho}, \texttt{gamma}, \texttt{beta}) over time to
\emph{constrain} the effects to be equal at all points in time. This
would be the default behaviour if we were specifying a random or fixed
effects model conventionally with the stacked long-format data, i.e.,
with the stacked data, only one of each coefficient is estimated over
all points in time. We can easily relax this assumption of constant
effects by changing the labels to be different over time (e.g.,
\texttt{rhot1}, \texttt{rhot2}, etc.), or we could just delete them
completely to have each estimated separately.

Lastly, the covariances or correlations are specified in lines 21--28.
The thing that makes this model a fixed effects model, rather than a
random effects one, is the assumption of the relatedness of the
individual effects. Specifically, we allow \texttt{alpha} to correlate
with \texttt{eta} to account for

\begin{Shaded}
\begin{Highlighting}[numbers=left,,]
\NormalTok{m1 \textless{}{-}}\StringTok{ \textquotesingle{}}
\StringTok{\# Measurement models }
\StringTok{xt1 =\textasciitilde{} x11 + x21 + x31}
\StringTok{xt2 =\textasciitilde{} x12 + x22 + x32}
\StringTok{xt3 =\textasciitilde{} x13 + x23 + x33}
\StringTok{xt4 =\textasciitilde{} x14 + x24 + x34}
\StringTok{yt1 =\textasciitilde{} y11 + y21 + y31}
\StringTok{yt2 =\textasciitilde{} y12 + y22 + y32}
\StringTok{yt3 =\textasciitilde{} y13 + y23 + y33}
\StringTok{yt4 =\textasciitilde{} y14 + y24 + y34}
\StringTok{\# Individual effects}
\StringTok{alpha =\textasciitilde{} 1*xt2 + 1*xt3 + 1*xt4 }
\StringTok{eta   =\textasciitilde{} 1*yt2 + 1*yt3 + 1*yt4}
\StringTok{\# Regressions, time{-}invariant effects}
\StringTok{xt2 \textasciitilde{} phi*xt1 + gamma*yt1 }
\StringTok{xt3 \textasciitilde{} phi*xt2 + gamma*yt2}
\StringTok{xt4 \textasciitilde{} phi*xt3 + gamma*yt3}
\StringTok{yt2 \textasciitilde{} rho*yt1 + beta*xt1 }
\StringTok{yt3 \textasciitilde{} rho*yt2 + beta*xt2}
\StringTok{yt4 \textasciitilde{} rho*yt3 + beta*xt3}
\StringTok{\# Correlations}
\StringTok{alpha \textasciitilde{}\textasciitilde{} eta + xt1 + yt1}
\StringTok{eta   \textasciitilde{}\textasciitilde{} xt1 + yt1 }
\StringTok{xt1   \textasciitilde{}\textasciitilde{} yt1}
\StringTok{\# Contemporary residual correlations}
\StringTok{xt2 \textasciitilde{}\textasciitilde{} yt2}
\StringTok{xt3 \textasciitilde{}\textasciitilde{} yt3}
\StringTok{xt4 \textasciitilde{}\textasciitilde{} yt4}
\StringTok{\textquotesingle{}}
\end{Highlighting}
\end{Shaded}

\begin{Shaded}
\begin{Highlighting}[]
\NormalTok{m1.fit \textless{}{-}}\StringTok{ }\KeywordTok{sem}\NormalTok{(}\DataTypeTok{model =}\NormalTok{ m1, }\DataTypeTok{data =}\NormalTok{ df1, }\DataTypeTok{meanstructure =} \OtherTok{TRUE}\NormalTok{, }\DataTypeTok{estimator =} \StringTok{"ML"}\NormalTok{)}
\KeywordTok{summary}\NormalTok{(m1.fit, }\DataTypeTok{fit.measures =} \OtherTok{TRUE}\NormalTok{, }\DataTypeTok{standardized =} \OtherTok{TRUE}\NormalTok{)}
\end{Highlighting}
\end{Shaded}

The \emph{configural} measurement invariance model allows (1) the factor
loadings in the measurement models

\hypertarget{one-sided-models}{%
\subsubsection{One-sided models}\label{one-sided-models}}

We can conceive of one-sided versions of the same DPM and ALT \citep[see
for example][]{Allison2017}. In these models, we treat \(y_{t}\) as the
`definitive' dependent variable\footnote{This is an arbitrary choice,
  however. We could just as well treat \(x_{t}\) as the dependent
  variable.} and leave the reciprocal effects
\(y_{t-1} \rightarrow x_{t}\) unmodelled. Now, \(x_{t}\) is an exogenous
variable, allowed to correlate with \(x_{s}, \ t \ne s\), as well as the
predetermined \(y_{0}\) and the individual effects \(\eta\). This is
obviously a more general model that the two-sided version that places
less assumptions on the `independent' variable. However, it is important
to note that if we believe that \(x_{t}\) is a function of \(y_{t-1}\),
we must model this dependency even if we are only truly interested in
the effect \(x_{t-1} \rightarrow y_{t}\). To see why this is, write
\(x_{it} = \varphi x_{it-1} + \gamma y_{it-1} + \upsilon_{it}\), where
\(\upsilon_{it} = \alpha_{i} + \delta_{it}\). Now, expand for
\(x_{it} = \varphi x_{it-1} + \gamma (\rho y_{t-2} + \beta x_{t-2} + \eta_{i} + \varepsilon_{it-1}) + \upsilon_{it}\).
Obviously, because of the assumed presence of the reciprocal effect,
\(x_{t}\) will be correlated with \(\varepsilon_{it-1}\). Thus, we must
treat \(x_{t}\) as \emph{sequentially exogenous} and allow for
\(Cov(x_{t},\varepsilon_{s}), \ t > s\).

\hypertarget{residual-level-models}{%
\subsection{Residual-level models}\label{residual-level-models}}

The RI-CLPM and LCM-SR are considered `residual-level' models in which
the autoregressive and covariate effects are specified between the
residuals of the observed variables or the disturbances of the latent
variables representing the measurement error adjusted constructs of
interest, see Figure xxx(c) and xxx(d) for two-sided representations.

For an RI-CLPM analogous to the two-sided DPM above, we have
\begin{align}
\begin{split}
y_{it} & = \eta_{i} + \varepsilon_{it}, \\
x_{it} & = \alpha_{i} + \delta_{it}, \ t = 0, \ldots, T, \\
\varepsilon_{it} & = \rho \varepsilon_{it-1} + \beta \delta_{it-1} + \nu_{it}, \\
\delta_{it} & = \varphi \delta_{it-1} + \gamma \varepsilon_{it-1} + \upsilon_{it}, \ t = 1, \ldots, T, \label{riclpm}
\end{split}
\end{align} where \(\varepsilon_{it}\) and \(\delta_{it}\) are the
errors or disturbances of either the observed or latent variables
\(y_{it}\) and \(x_{it}\), respectively. If the variables are modelled
as latent, then the measurement models in Equation \eqref{meas-model}
apply here as well. Note that the autoregressive and cross-lagged paths
are specified between these `structured residuals' as \citet{Curran2014}
refer to them. Note further that the factor loadings of the individual
effects on the observed or latent variables of interest, \(y_{it}\) and
\(x_{it}\), are fixed to one at all points in time,
\(t = 0, \ldots, T\). This makes the default residual-level models as
they are described in the source articles from \citet{Curran2014};
\citet{Hamaker2015} implicitly constrained versions of their
observation-level counterparts, see \citet{Andersen2021};
\citet{Ou2016}; \citet{Hsiao2014} for details. This means that the
residual-level models will not generally be equivalent to the
predetermined observation-level counterparts. If the assumptions of the
constrained models hold, i.e., stationarity and equilibrium, then the
observed covariances \(Cov(y_{0},\eta)\) and \(Cov(x_{0},\alpha)\) will
equal the constraints placed on the constrained models (barring sampling
error) and the estimated autoregressive and cross-lagged coefficients
will be roughly equal. However, if the assumptions hold, the
residual-level models, like their constrained observation-level
counterparts, are more parsimonious, fixing the initial factor loadings
\(\eta \rightarrow y_{0}\) and \(\alpha \rightarrow x_{0}\) to the
appropriate values without having to estimate them.

An analogous LCM-SR further incorporates random slopes per unit.
\begin{align}
\begin{split}
y_{it} & = \eta_{1i} + t\eta_{2i} + \varepsilon_{it}, \\
x_{it} & = \alpha_{1i} + t\alpha_{2i} + \delta_{it}, \ t = 0, \ldots, T, \\
\varepsilon_{it} & = \rho \varepsilon_{it-1} + \beta \delta_{it-1} + \nu_{it}, \\
\delta_{it} & = \varphi \delta_{it-1} + \gamma \varepsilon_{it-1} + \upsilon_{it}, \ t = 1, \ldots, T, \label{riclpm}
\end{split}
\end{align} where \(\eta_{2i}\) and \(\alpha_{2i}\) are the random
slopes for \(y_{it}\) and \(x_{it}\), respectively.

\hypertarget{xxx}{%
\subsubsection{xxx}\label{xxx}}

WHAT ABOUT FIXED VS. RANDOM EFFECTS IN THE TWO-SIDED MODELS? IS IT
ENOUGH TO ALLOW ALPHA AND ETA TO COVARY? TEST THIS!

\hypertarget{conclusion}{%
\section{Conclusion}\label{conclusion}}

xxx

\hypertarget{acknowledgements}{%
\section*{Acknowledgement(s)}\label{acknowledgements}}
\addcontentsline{toc}{section}{Acknowledgement(s)}

An unnumbered section,
e.g.~\texttt{\textbackslash{}section*\{Acknowledgements\}}, may be used
for thanks, etc.~if required and included \emph{in the non-anonymous
version} before any Notes or References.

\hypertarget{disclosure-statement}{%
\section*{Disclosure statement}\label{disclosure-statement}}
\addcontentsline{toc}{section}{Disclosure statement}

An unnumbered section,
e.g.~\texttt{\textbackslash{}section*\{Disclosure\ statement\}}, may be
used to declare any potential conflict of interest and included \emph{in
the non-anonymous version} before any Notes or References, after any
Acknowledgements and before any Funding information.

\hypertarget{funding}{%
\section*{Funding}\label{funding}}
\addcontentsline{toc}{section}{Funding}

An unnumbered section,
e.g.~\texttt{\textbackslash{}section*\{Funding\}}, may be used for grant
details, etc.~if required and included \emph{in the non-anonymous
version} before any Notes or References.

\hypertarget{notes-on-contributors}{%
\section*{Notes on contributor(s)}\label{notes-on-contributors}}
\addcontentsline{toc}{section}{Notes on contributor(s)}

An unnumbered section,
e.g.~\texttt{\textbackslash{}section*\{Notes\ on\ contributors\}}, may
be included \emph{in the non-anonymous version} if required. A
photograph may be added if requested.

\hypertarget{notes}{%
\section*{Notes}\label{notes}}
\addcontentsline{toc}{section}{Notes}

An unnumbered \texttt{Notes} section may be included before the
References (if using the \texttt{endnotes} package, use the command
\texttt{\textbackslash{}theendnotes} where the notes are to appear,
instead of creating a \texttt{\textbackslash{}section*}).

\bibliographystyle{tfcad}
\bibliography{r-references.bib}


\input{"appendix.tex"}


\end{document}
